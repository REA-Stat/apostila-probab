
% !TeX spellcheck = pt_BR
% !TEX encoding = UTF-8 Unicode

% Ver copyright.tex para direitos autorais e licença.

\clearpage
\section{Espaços de Probabilidade}

Na seção \ref{espaços de probabilidade uniforme}, definimos um espaço de probabilidade uniforme como o triplete $(\Omega,\mathcal{F},\Pb)$, onde $\Omega$ é um conjunto finito, $\mathcal{F}=\mathcal{P}(\Omega)$ e $\Pb:\mathcal{F}\to[0,1]$ satisfaz as propriedades \eqref{P_prop1} - \eqref{P_prop3}. Agora, vamos generalizar o conceito de espaço de probabilidade para permitir o seguinte:
\begin{itemize}
    \item Espaços amostrais arbitrários $\Omega$.
    \item Espaços de eventos que refletem informações parciais - não é necessário, e às vezes nem é possível, que $\mathcal{F} = \mathcal{P}(\Omega)$.
    \item Probabilidade definida em um espaço de eventos geral.
\end{itemize}

\subsection{Espaço amostral e espaço de eventos}

\begin{definition}
    Um {\it espaço amostral} $\Omega$ é o conjunto de todos os possíveis resultados de um processo aleatório (ou experimento), ou seja, um processo cujo resultado não pode ser determinado antecipadamente. Pode ser qualquer conjunto.
\end{definition}

\begin{example}
Qual é o espaço amostral correspondente aos seguintes processos?
\begin{itemize}
    \item O lançamento de uma moeda: $\Omega=\{Cara, Coroa\}$.
    \item O lançamento de um dado: $\Omega=\{1,2,3,4,5,6\}$.
    \item O número de e-mails enviados por um endereço @google.com em um ano: $\Omega=\N$.
    \item O peso de uma maçã: $\Omega=[0,1]$.
    \item A posição de um dardo lançado em um tabuleiro quadrado de tamanho 1: $\Omega=[0,1]\times[0,1]$.
    \item O preço das ações do Twitter em um ano: $\Omega=\R$.
    \item As flutuações de temperatura em Coventry em 2023: neste caso, o espaço amostral é uma função completa, mapeando o tempo $t$ para um número em $[-50,50]$.
    \item O estado do mundo em um ano! Neste caso, o espaço amostral não pode ser descrito, mas existe como conceito e pode ser parcialmente observado por meio de sua interação com processos que podem ser medidos (por exemplo, afetará as taxas de juros ou o número de internações hospitalares em um determinado dia no futuro, a frequência de eventos climáticos extremos, etc).
\end{itemize}
\end{example}

\begin{remark}
   Embora $\Omega$ possa ser qualquer conjunto em teoria, em ST120, consideraremos apenas os casos em que a cardinalidade de $\Omega$ é igual a $n$, para algum $n\in\N$ (espaço amostral finito), $|\Omega|=|\N|$ (espaço amostral contável) ou $|\Omega|=|\R|$ (espaço amostral não contável ou contínuo - para incluir intervalos ou produtos cartesianos de $\R$ e seus intervalos).
\end{remark}

Como já vimos no caso do espaço de probabilidade uniforme, eventos são subconjuntos de $\Omega$. Mais geralmente, um evento é um subconjunto de $\Omega$ quando é possível dizer se algum resultado dado pertence ao conjunto (ou seja, 'o evento ocorreu') ou não, dadas as informações que temos sobre o resultado - observe que nem sempre temos informações completas sobre o resultado e o espaço de eventos reflete as informações que temos. O exemplo a seguir demonstra exatamente essa propriedade do espaço de eventos.

\begin{example}
\label{ex: eventos e informações}
Suponha que eu lance um dado ($\Omega = \{1,2,3,4,5,6\}$) e relato as seguintes informações a dois alunos: eu digo a James se o resultado é um número par ou não, e digo a Lily o quociente de $(\omega-1)$ dividido por dois (ou seja, relato $0$ para $\{1,2\}$, $1$ para $\{3,4\}$ e $2$ para $\{5,6\}$). Quais são os espaços de eventos correspondentes?

James só sabe se o resultado é ímpar ou par, então ele só pode dizer se pertence a $\{1,3,5\}$ ou $\{2,4,6\}$. Já que, pela construção de $\Omega$, todos os resultados estão em $\Omega$, ele também pode dizer que o resultado está em $\Omega$ e não em $\emptyset$ - observe que tanto $\emptyset$ quanto $\Omega$ são subconjuntos de $\Omega$. Portanto, a coleção de eventos (ou seja, o espaço de eventos) correspondente às informações que James possui é 
\[
\mathcal{F}_J = \left\{ \emptyset, \{1,3,5\}, \{2,4,6\}, \Omega \right\}. 
\]
(Pense sobre o motivo pelo qual James não pode afirmar com certeza se o resultado está em outro subconjunto).

Com base nas informações fornecidas a ela, Lily poderá dizer se o resultado está em $\{1,2\}, \{3,4\}$ ou $\{5,6\}$. Ela também pode dizer se o resultado estará em $\{1,2,3,4\}, \{1,2,5,6\}$ ou $\{3,4,5,6\}$ (por exemplo, $\{1,2,3,4\}$ corresponde ao número informado a Lily sendo $0$ ou $1$) e ela pode dizer, por padrão, que o resultado estará em $\Omega$ e não em $\emptyset$ (o $\emptyset$ é definido como o complemento de $\Omega$ em relação a $\Omega$, então todos os pontos em $\Omega$ que não estão em $\Omega$, que é, é claro, nenhum! É mais fácil pensar nisso como o evento em que o resultado não está no espaço amostral do que no evento em que nada acontece). Portanto, o espaço de eventos correspondente às informações fornecidas a Lily é
\[
\mathcal{F}_L = \left\{ \emptyset, \{1,2\}, \{3,4\}, \{5,6\},\{1,2,3,4\},\{1,2,5,6\},\{3,4,5,6\},\Omega \right\}. 
\]
Observe que há redundância na forma como as informações são codificadas no espaço de eventos - a informação corresponde a saber simultaneamente a resposta para 'o evento ocorreu ou não' para todos os eventos no espaço de eventos, mas saber, por exemplo, que tanto $\{1,2\}$ quanto $\{1,2,3,4\}$ aconteceram nos permite deduzir a resposta para tudo o mais. Que suposição estamos fazendo que nos permite dizer isso?
\end{example}

Se $A$ e $B$ são eventos, de acordo com nossa intuição, esperamos que o seguinte também sejam eventos:
\begin{itemize}[noitemsep]
\item $A \cap B$ (Ambos $A$ e $B$ aconteceram).
\item $A \cup B$ (Ou $A$ ou $B$ aconteceu).
\item $A^c$ ($A$ não aconteceu).
\item $A \setminus B=A \cap B^c$ ($A$ aconteceu, mas $B$ não aconteceu).
\end{itemize}

\begin{notation*}
Quando escrevemos $A^c$, implicitamente estamos considerando o complemento em relação a um espaço amostral dado $ \Omega $.
Uma notação mais explícita é escrever $ \Omega \setminus A $.
\end{notation*}

Portanto, gostaríamos que o espaço de eventos fosse fechado sob as operações de união, interseção, complemento e diferença (quando dizemos que um conjunto é fechado sob uma operação, queremos dizer que, se aplicarmos a operação a quaisquer elementos do conjunto, o resultado ainda estará no conjunto). Será que isso é suficiente? Vamos considerar o seguinte
\begin{example}
Considere o caso em que $\Omega=\N$ (ou seja, qualquer número natural pode ser o resultado do processo aleatório que consideramos) e suponha que temos informações suficientes para dizer se o evento $\{n\}$ aconteceu ou não, para qualquer $n\in\N$. De acordo com nossa intuição, se podemos dizer se qualquer resultado $\omega$ pertence a qualquer conjunto $\{n\}$ ou não (o que você pode dizer para $\omega$ quando $\omega\in\{n\}$?), então também deveríamos ser capazes de dizer se $\omega\in\{2n|n\in\N\}$ ou não, ou seja, deveríamos ser capazes de dizer se $\omega$ é par. Portanto, esperamos
\[
\bigcup_{n=0}^\infty \{2n\} = \{2n|n\in\N\}
\]
estar no espaço de eventos. Agora estamos fazendo a suposição de que o espaço de eventos não é apenas fechado para uniões, mas também para uniões de conjuntos contáveis (ou seja, infinitos, mas com cardinalidade igual a $|\N|$).
\end{example}

Seguindo nossa intuição, definimos o espaço de eventos da seguinte forma

\begin{definition} \label{def: espaço de eventos}
Seja $\Omega$ o espaço amostral e $\cF$ uma coleção de subconjuntos de $\Omega$. $\cF$ é um \emph{espaço de eventos} (também chamado de $\sigma$-\emph{álgebra}) se satisfaz as seguintes condições:
\begin{enumerate}
[$ (i) $]
\item $\Omega\in\cF$.
\item se $A \subseteq \Omega$, $A \in \cF \Rightarrow A^c \in \cF$ ($\cF$ é fechado sob complementos).
\item se $\{A_n: n \in \N\}$ é tal que $A_n \in \cF$ $\forall \, n$, então
\begin{equation}
\bigcup_{n=1}^{\infty} A_n \in \cF \,.
\end{equation}
($\cF$ é fechado sob uniões contáveis.)
\end{enumerate}
\end{definition}

\begin{exercise}
Seja $\Omega$ um conjunto não vazio e $\cF=\mathcal{P}(\Omega)$, ou seja, o conjunto de todos os subconjuntos de $\Omega$. Então, $\cF$ é um espaço de eventos em $\Omega$. 
\end{exercise}

\begin{solution}
Precisamos mostrar que $\cF=\mathcal{P}(\Omega)$ satisfaz as três propriedades da \ref{def: espaço de eventos}.
\begin{enumerate}
[$ (i) $]
\item $\Omega \subseteq \Omega$ e, portanto, $\Omega\in\mathcal{P}(\Omega) = \cF$.
\item Suponha que $A \subseteq \Omega$, $A \in \cF$. Então, $A^c = \Omega \setminus A \subseteq \Omega$. Assim, $A^c \in \mathcal{P}(\Omega) =\cF$.
\item Suponha que $A_n$ são tais que $A_n \subseteq \Omega$, $\forall n\in\N$. Então
\begin{equation}
\bigcup_{n=1}^{\infty}A_n \subseteq \bigcup_{n=1}^{\infty} \Omega = \Omega
\Rightarrow \bigcup_{n=1}^{\infty}A_n \in \mathcal{P}(\Omega) =\cF \,.
\end{equation}
\end{enumerate}
\end{solution}

\begin{exercise}

Seja $A \subseteq \Omega$ um subconjunto não vazio de $\Omega$. Então
\begin{equation}
\{ \emptyset, A, A^c, \Omega\}
\end{equation}
é um espaço de eventos em $\Omega$.
\end{exercise}

\begin{solution}
Precisamos mostrar que $\{ \emptyset, A, A^c, \Omega\}$ satisfaz as três propriedades da \ref{def: espaço de eventos}.
\begin{enumerate}
[$ (i) $]
\item $\Omega\in\cF$ por definição.
\item Verificamos que a propriedade é satisfeita para cada evento:  $\emptyset^c=\Omega \in \cF$, $A^c \in \cF$, $(A^c)^c=A \in \cF$,
$\Omega^c=\emptyset \in \cF$.
\item Seja $\{B_n:n\in \N\}$ uma sequência de subconjuntos de $\Omega$ tal que $B_n \in \cF$ $\forall \, n$. Vamos considerar todas as possibilidades:
\begin{itemize}
\item Se todos os conjuntos $B_n$ forem idênticos e iguais a $B$ (ou seja, $B\in\cF$), ou existe uma subsequência $B_{n_k}$ de conjuntos que são idênticos a $B$ e o restante é todo $\emptyset$, então
$\bigcup_{n=1}^\infty B_n = B\in \cF$.
\item Se pelo menos um dos conjuntos for o espaço amostral (por exemplo, $B_i = \Omega$, para algum $i\in\N$), então
\[
\bigcup_{n=1}^\infty B_n = \Omega
\]
e, portanto, $\bigcup_{n=1}^\infty B_n \in \cF$. 
\item Se houver pelo menos um conjunto $A$ e um conjunto $A^c$, então
\[ \Omega = A\cup A^c \subseteq \bigcup_{n=1}^\infty B_n \subseteq \Omega,\]
onde o último relacionamento segue do fato de que todos os eventos são subconjuntos de $\Omega$. Portanto, $\bigcup_{n=1}^\infty B_n = \Omega$ e, assim, $\bigcup_{n=1}^\infty B_n\in \cF$.
\end{itemize}
\end{enumerate}
\end{solution}

\begin{proposition} \label{prop_sigma_prop}
Seja $\cF$ um espaço de eventos em $\Omega$. Então
\begin{enumerate}
\item $\cF$ é fechado sob uniões finitas.
\item $\cF$ é fechado sob interseções finitas.
\item $\cF$ é fechado sob interseções contáveis.
\end{enumerate}
\end{proposition}

\begin{proof}
~
\begin{enumerate}
\item Sejam $A_1, \dots, A_n \in \cF$. Defina $A_j = \emptyset\in\cF \,\, \forall \,\, j>n$, portanto $A_n \in \cF \, \forall \, n\geq 1$. Como os conjuntos vazios não contribuirão para a união, podemos mostrar que
\[
\bigcup_{j=1}^{n} A_j = \bigcup_{j=1}^\infty A_j\in\cF,
\]
pois $\cF$ é fechado sob uniões contáveis.
\item Sejam $A_1, \dots, A_n \in \cF$. Queremos mostrar que $\bigcap_{j=1}^{n}A_j \in \cF$. Pela lei de De Morgan (mostrando que cada elemento de um conjunto precisa pertencer ao outro conjunto também)
\[
\bigcap_{j=1}^{n}A_j = \left(\bigcup_{j=1}^{n}A_j^c\right)^c.
\]
Como o espaço de eventos $\cF$ é fechado sob complementos, $A_j^c\in \cF$, para todo $j=1,\dots,n$. Como é fechado sob uniões finitas (afirmação 1 da proposição, mostrada acima),
\[
\bigcup_{j=1}^{n}A_j^c \ \in \cF
\]
Ao tomar o complemento mais uma vez, segue que $\bigcap_{j=1}^{n}A_j\in\cF$.
\item A prova é semelhante à da afirmação 2 acima, observando que a lei de De Morgan também vale para uniões e interseções contáveis. Ou seja, podemos escrever
\[
\bigcap_{j=1}^{\infty}A_j = \left(\bigcup_{j=1}^{\infty}A_j^c\right)^c.
\]
\end{enumerate}
\end{proof}

\subsection{Probabilidade}

O último elemento da tripla do espaço de probabilidade é a medida de probabilidade $\Pb$.
Na definição \ref{def:espaço de probabilidade uniforme} do espaço de probabilidade uniforme, definimos a medida de probabilidade como uma função do espaço de eventos para $[0,1]$, de modo que a probabilidade do evento $\Omega$ (`o resultado está no espaço amostral') é $1$ e para dois eventos disjuntos $A,B$, $\Pb(A\cup B)=\Pb(A)+\Pb(B)$ -- a última propriedade pode ser generalizada por indução para a aditividade finita: se $A_1,\dots,A_n$ são eventos disjuntos, então
\[
\Pb(\bigcup_{i=1}^n A_k) = \sum_{k=1}^n \Pb(A_k).
\]
Isso é suficiente para espaços de probabilidade infinitos? Vamos considerar o seguinte
\begin{example}
Seja $\Omega=\N^*=\{1,2,\dots\}$ o conjunto dos números naturais positivos e $\cF=\mathcal{P}(\Omega)$. Suponha que $\Pb(\{n\}) = \frac{1}{2^n}$, para todo $n\geq 1$. O que esperaríamos que o evento $\{ 2n|n\geq 1\}$ (`o resultado é um número par') seja?

Intuitivamente, somaríamos as probabilidades correspondentes ao resultado ser par, ou seja,
\[
\Pb(\{2n|n\geq 1\}) = \sum_{n=1}^\infty \Pb(\{2n\}) = \sum_{n=1}^\infty \frac{1}{2^{2n}} = \sum_{n=1}^\infty \frac{1}{4^n} = \frac{1}{3}.
\]

(Note que o evento $\{n\}$ corresponde a `o resultado é $n$'). O cálculo acima não pode ser justificado, a menos que estendamos a propriedade da aditividade finita para valer também para uniões contáveis de eventos disjuntos. De fato, é isso que fazemos!
\end{example}

\begin{definition}
[Medida de Probabilidade]
\label{def:medidadeprob}
Dado um espaço amostral $ \Omega $ e um espaço de eventos $ \cF $, uma função $ \Pb:\cF\to\R $ é chamada de \emph{medida de probabilidade} se satisfaz
        \begin{enumerate}[$ (i) $]
	    \item $ \Pb(B) \in [0,1] $ para todo $ B \in \mathcal{F} $;
            \item $\Pb(\Omega)=1$;
            \item (Aditividade Contável) Para todo $A_n\in\cF,\ n\geq 1$ eventos disjuntos (ou seja, para todos $m,n\geq$ tais que $m\neq n$, $A_m\cap A_n = \emptyset$, 
            \begin{equation}
            \label{property: aditividadecountable}
                \Pb\left(\bigcup_{n=1}^\infty A_n \right) = \sum_{n=1}^\infty \Pb(A_n).
            \end{equation}
        \end{enumerate}
\end{definition}

Nós agora apresentamos a definição de um espaço de probabilidade abstrato.
\begin{definition}
\label{def: probability space}
Um \emph{espaço de probabilidade} é definido como o triplo $(\Omega, \mathcal{F}, \Pb)$, onde
    \begin{itemize}
        \item $\Omega$ (o \emph{espaço amostral}) é o conjunto de todos os possíveis resultados do experimento (sempre assumimos que não é vazio);
        \item $\mathcal{F}$ é um espaço de eventos de subconjuntos de $\Omega$. 
        \item $\Pb$ é uma medida de probabilidade em $ \cF $.
    \end{itemize}
\end{definition}

\begin{proposition}
\label{prop: basic properties of probability}
Seja $(\Omega, \mathcal{F}, \Pb)$ um espaço de probabilidade. Então, $\Pb$ possui as seguintes propriedades
\begin{enumerate}
    \item Se $A, B \in \mathcal{F}$ tal que $A \subseteq B$, então 
    \[ \Pb(B - A) = \Pb(B) - \Pb(A).\]
    Observe que $B - A = B \cap A^c$ e deve ser interpretado como 'todos os elementos de $B$ que não estão em $A'$.
    \item Para todo $A \in \mathcal{F}$,
    \[ \Pb(A^c) = 1 - \Pb(A).\]
    \item $\Pb(\emptyset) = 0$.
\end{enumerate}
\end{proposition}

\begin{proof}
~
    \begin{enumerate}
        \item Escrevemos $B$ como a união do conjunto com todos os elementos em $B$ que não estão em $A$ e aqueles que estão, ou seja, $B = (B - A) \cup A$. Pela aditividade finita,
        \[ \Pb(B) = \Pb((B - A) \cup A) = \Pb(B - A) + \Pb(A),\]
        o que prova a afirmação.
        \item Usando a propriedade acima,
        \[\Pb(A^c) = \Pb(\Omega - A) = \Pb(\Omega) - \Pb(A) = 1 - \Pb(A).
        \]
    \item $\Pb(\emptyset) = \Pb(\Omega^c) = 1 - \Pb(\Omega) = 1 - 1 = 0.$
    \end{enumerate}
\end{proof}

Podemos usar a aditividade contável para calcular a probabilidade de uma união de eventos disjuntos. Como podemos calcular a probabilidade de qualquer união de eventos? A seguinte proposição nos fornece uma maneira de fazer isso.

\begin{proposition}[Fórmula de Inclusão-Exclusão]
Seja $(\Omega, \mathcal{F}, \Pb)$ um espaço de probabilidade. Então, para qualquer coleção finita $A_1, \dots, A_n$ de eventos em $\mathcal{F}$, temos
\begin{equation}
\label{eq: inclusion-exclusion}
\Pb\left( \bigcup_{k=1}^{n} A_k \right) = \sum_{k=1}^{n}(-1)^{k-1} 
\sum_{1 \leq i_1 < \dots < i_k \leq n} \Pb(A_{i_1} \cap \dots \cap A_{i_k}) \,.
\end{equation}
\end{proposition}

\begin{remark}
A fórmula \ref{eq: inclusion-exclusion} acima usa uma notação concisa e não é imediatamente fácil de interpretar. Para entendê-la melhor, vamos considerar alguns casos específicos.
\begin{itemize}

\item
$n=2$
\begin{align}
\Pb\left(\bigcup_{k=1}^{2} A_k\right) &= \sum_{k=1}^{2} (-1)^{k-1} \sum_{1 \leq i_1 < \dots < i_k \leq 2} \Pb(A_1 \cap \dots \cap A_k) \\
&= \sum_{1 \leq i \leq 2} \Pb(A_i) - \sum_{1 \leq i_1 < i_2 \leq 2} \Pb(A_{i_1} \cap A_{i_2}) \\
&= \Pb(A_1) + \Pb(A_2) - \Pb(A_1 \cap A_2) \,.
\end{align}
\item
$n=3$
\begin{align}
\Pb\left(\bigcup_{k=1}^{3} A_k\right) &= \sum_{k=1}^{3} (-1)^{k-1} \sum_{1 \leq i_1 < \dots < i_k \leq 3} \Pb(A_1 \cap \dots \cap A_k) \\
&= \sum_{1 \leq i \leq 3} \Pb(A_i) - \sum_{1 \leq i_1 < i_2 \leq 3} \Pb(A_{i_1} \cap A_{i_2}) \\
&\hspace{0.4cm} + \sum_{1 \leq i_1 < i_2 < i_3 \leq 3} \Pb(A_{i_1} \cap A_{i_2} \cap A_{i_3})\\
&= \Pb(A_1) + \Pb(A_2) + \Pb(A_3) - \Pb(A_1 \cap A_2) - \Pb(A_1 \cap A_3) \\
&\hspace{0.4cm} - \Pb(A_2 \cap A_3) + \Pb(A_1 \cap A_2 \cap A_3)\,.
\end{align}
Portanto, a soma $\sum_{1 \leq i_1 < \dots < i_k \leq 2}$ deve ser interpretada como a soma de todos os $k$-uplas $(i_1, \dots, i_k)$ de números $\{1, \dots, n\}$ sem repetição (as desigualdades são estritas). Como vimos na seção \ref{sec: how to count}, existem $\binom{n}{k}$ dessas $k$-uplas, portanto a soma terá $\binom{n}{k}$ parcelas.
\end{itemize}
\end{remark}

\begin{proof}
Vamos provar o resultado apenas para $n=2$ (a prova do passo de indução no caso geral é semelhante, mas mais confusa!). Escrevemos
\[
A_1 \cup A_2 = (A_1-B) \cup B \cup (A_2-B),
\]
onde $B=A_1 \cap A_2$. Os conjuntos $A_1-B$, $B$ e $A_2-B$ são todos disjuntos, então podemos escrever
\[
\Pb(A_1 \cup A_2) = \Pb\left( (A_1-B) \cup B \cup (A_2-B) \right) = \Pb(A_1-B) + \Pb(B) + \Pb(A_2-B)
\]
usando a aditividade finita. Sabemos, pela proposição \ref{prop: basic properties of probability}, que $\Pb(A_1-B) = \Pb(A_1) - \Pb(B)$ e, da mesma forma, $\Pb(A_2-B) = \Pb(A_2) - \Pb(B)$. Substituindo esses valores na fórmula acima, obtemos
\[
\Pb(A_1 \cup A_2) = (\Pb(A_1) - \Pb(B)) + \Pb(B) + (\Pb(A_2) - \Pb(B)) = \Pb(A_1) + \Pb(A_2) - \Pb(B),
\]
o que comprova a alegação.
\end{proof}

\begin{proposition} \label{prop_prob}
Seja $(\Omega, \cF, \Pb)$ um espaço de probabilidade. Se $A, B \in \cF$ e $A \subseteq B$, então
\begin{equation}
\Pb(A) \leq \Pb(B) \,.
\end{equation}
\end{proposition}

\begin{proof}  
Uma vez que $A \subseteq B$, segue que $\Pb(B-A) = \Pb(B) - \Pb(A)$ ou, equivalentemente, $\Pb(A) = \Pb(B) - \Pb(B-A) \leq \Pb(B)$, onde a desigualdade decorre do fato de que as probabilidades são sempre não negativas.
\end{proof}

\begin{proposition}
[Desigualdade de Boole]
Seja $(\Omega, \cF, \Pb)$ um espaço de probabilidade. Se $A_1, \dots, A_n \in \cF$, então 
\begin{equation}
\Pb\left( \bigcup_{i=1}^{n} A_i \right) \leq \sum_{i=1}^{n} \Pb(A_i) \quad {\rm (*)}
\end{equation}
\end{proposition}

\begin{proof}
Procedemos por indução. Para $n=2$, observe que 
\begin{equation}
\Pb(A_1 \cup A_2) = \Pb(A_1) + \Pb(A_2) - \underbrace{\Pb(A_1 \cap A_2)}_{\geq 0} \leq \Pb(A_1) + \Pb(A_2) \,.
\end{equation}
Portanto, (*) vale para $n=2$. Suponha agora que (*) vale $\forall \, j \leq n$, então precisamos provar que vale para $n+1$ eventos. Seja $A_1, \dots, A_{n+1} \in \cF$, então, argumentando como acima, temos
\begin{align}
\Pb\left(\bigcup_{j=1}^{n+1} A_j\right) &= \Pb\left(\left(\bigcup_{j=1}^{n} A_j\right) \cup A_{n+1}\right) \\
&= \Pb\left(\bigcup_{j=1}^{n} A_j\right) + \Pb\left(A_{n+1}\right) - \Pb\left(\left(\bigcup_{j=1}^{n} A_j\right) \cap A_{n+1}\right)\\
&\leq \Pb\left(\bigcup_{j=1}^{n} A_j\right) + \Pb(A_{n+1}) \\
&\leq \sum_{j=1}^{n} \Pb(A_j) + \Pb(A_{n+1}) = \sum_{j=1}^{n+1} \Pb(A_j) \,.
\qedhere
\end{align}
\end{proof}

\subsection*{Revisão de espaços de probabilidade}

$ \Omega $ -- \emph{espaço amostral}: elementos $ \omega\in\Omega $ são resultados, $ \Omega $ é um conjunto não vazio.

$ \cF $ -- \emph{espaço de eventos}: elementos $ A \in \Omega $ são eventos ($ A \subseteq \Omega $)

Deve satisfazer três condições:
\begin{itemize}
\item
$\cF \ne \emptyset$
\item
$ A^c \in \cF $ para todo $ A \in \cF $
\item
$ (\cup_{n=1}^\infty A_n) \in \cF $ para toda sequência de eventos $ A_1,A_2,A_3,\dots $
\end{itemize}

Consequências dessas condições:
\begin{itemize}
\item
$ \Omega \in \cF $ (de fato, tome $ A \in \cF $, $ A \cup A^c = \Omega $)
\item
$ \emptyset \in \cF $ (de fato, $ \Omega^c = \emptyset $)
\item
$ (\cap_{n=1}^\infty A_n) \in \cF $
para cada sequência de eventos $ A_1, A_2, A_3, \dots $
\\
(de fato, $ \cap_{n=1}^\infty A_n = (\cup_{n=1}^\infty A_n^c)^c $)
\end{itemize}

$ \Pb $
--
\emph{medida de probabilidade}: $ \Pb:\cF \to \R $.

Deve satisfazer três condições:
\begin{itemize}
\item
$ \Pb(A) \geq 0 $ para todo $ A \in \cF $
\item
$ \Pb(\Omega)=1 $
\item
$ \Pb $ é aditiva contável:
$ \Pb(\cup_{n=1}^\infty A_n) = \sum_{n=1}^\infty \Pb(A_n) $
\\
para cada sequência de eventos \emph{disjuntos} $ A_1, A_2, A_3, \dots $
\end{itemize}

O triplo $ (\Omega,\cF,\Pb) $ é chamado de \emph{espaço de probabilidade}.

\begin{example}
[Espaços de probabilidade uniformes]
$ \Omega $ é um conjunto finito,
$ \cF = \cP(\Omega) $,
$ \Pb(A) = \frac{|A|}{|\Omega|} $.
\end{example}

\begin{example}
Não existe um espaço de probabilidade para modelar o experimento "escolher um número inteiro aleatoriamente".
Por mais que gostaríamos de dizer que um número inteiro $ X $ escolhido aleatoriamente será par com probabilidade $ \frac{1}{2} $ e o último dígito em sua representação decimal será $ 7 $ com probabilidade $ \frac{1}{10} $, não existe um espaço de probabilidade que possa modelar isso.
Mais precisamente, para $ \Omega = \Z $, $ \cF = \Pb(\Z) $, não existe uma medida de probabilidade $ \Pb:\cF\to \R $ tal que $ \Pb(\{j\})=\Pb(\{k\}) $ para todos $ j, k \in \Z $.
De fato,
se $ \Pb(\{j\})>0 $, então $ \Pb(\Z)=\sum{x\in\Z} \Pb(\{x\}) = +\infty $,
e
se $ \Pb(\{j\})=0 $, então $ \Pb(\Z)=\sum{x\in\Z} \Pb(\{x\}) = 0 $, e em ambos os casos $ \Pb $ viola os requisitos para ser uma medida de probabilidade, ou seja, $ \Pb(\Z)=1 $.
\end{example}
