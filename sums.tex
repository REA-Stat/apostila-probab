
% !TeX spellcheck = pt_BR
% !TEX encoding = UTF-8 Unicode

% Ver copyright.tex para direitos autorais e licença.

\clearpage

\appendix


\section{Somas úteis}

\begin{align*}
&
(a+b)^n = \sum_{j=0}^{n} \binom{n}{j} a^j b^{n-j}
 &&
a,b \in \R, n \in \N_0
\\
&
\sum_{n=0}^\infty x^n = \frac{1}{1-x}
 &&
0 < x < 1
\\
&
\sum_{n=0}^\infty n x^{n-1} = \frac{1}{(1-x)^2}
 &&
0 < x < 1
\\
&
\sum_{n=0}^\infty n (n-1) x^{n-2} = \frac{2}{(1-x)^3}
 &&
0 < x < 1
\\
&
\sum_{n=0}^\infty n (n-1) (n-2) x^{n-3} = \frac{3!}{(1-x)^4}
 &&
0 < x < 1
\\
&
\sum_{k=0}^\infty \frac{x^k}{k!} = e^x
 &&
x \in \R
\\
&
\sum_{k=1}^n k = \frac{n(n+1)}{2}
 &&
n \in \N
\\
&
\sum_{k=1}^n k^2 = \frac{n(n+1)(2n+1)}{6}
 &&
n \in \N
\end{align*}

As cinco primeiras são:
teorema binomial,
série geométrica,
derivada da série geométrica,
segunda derivada da série geométrica,
terceira derivada da série geométrica.
Acontece que é legítimo diferenciar uma série da forma $ \sum_n a_n x^n $ termo a termo, mas não estamos preocupados com os detalhes de por que isso é verdade.

A quinta é a chamada "série de Taylor" da função exponencial.
Para verificar se a fórmula faz sentido, observe que ambos os lados resultam em $ 1 $ para $ x=0 $ e cada lado é igual à sua própria derivada.
Esses dois fatos implicam que ambos os lados são iguais para todos os valores de $ x $, mas não estamos preocupados com os detalhes disso.

As duas últimas fórmulas, uma vez escritas, podem ser provadas por indução (suponha que sejam corretas para um certo valor de $ n $, mostre que são corretas para $ n+1 $).
Se você está curioso sobre como essas fórmulas surgiram, elas podem ser derivadas fazendo inicialmente um palpite educado de que devem ser representadas por polinômios um grau maior do que o termo somado e, em seguida, usando os dois ou três primeiros termos para escrever um sistema de equações para os coeficientes.

\clearpage
\section{Exponenciais superam polinômios}

Para todo $ x \geq 0 $ e $ n \in \N $,
\[
e^x
\geq
1 + x + \frac{x^2}{2} + \frac{x^3}{3!} + \dots + \frac{x^n}{n!}
\]
\begin{proof}
Para $ n=0 $, já sabemos que $ e^x \geq 1 $.
Para $ n=1 $,
\[
e^x
=
1 + \int_0^x e^x \, \dd x
\geq
1 + \int_0^x 1 \, \dd x
=
1+x
\]
Para $ n=2 $,
\[
e^x
=
1 + \int_0^x e^x \, \dd x
\geq
1 + \int_0^x (1+x) \dd x
=
1+x+\tfrac{x^2}{2}
\]
Para $ n=3 $,
\[
e^x
=
1 + \int_0^x e^x \, \dd x
\geq
1 + \int_0^x (1+x+\tfrac{x^2}{2}) \dd x
=
1+x+\tfrac{x^2}{2}+\tfrac{x^3}{3!}
.
\]
O padrão é claro.
\end{proof}

Isso implica que
\[
\frac{a_0 + a_1 x + \dots + a_n x^n}{e^{ax}}
\]
tende a $ 0 $ à medida que $ x \to +\infty $, para todo $ a>0 $.

\begin{proof}
De fato, como $ e^{ax} \geq \frac{a^{n+1}}{(n+1)!} x^{n+1} $, cada termo em
\[
\frac{a_0}{e^{ax}}
+
\frac{a_1 x}{e^{ax}}
+ \dots +
\frac{a_n x^n}{e^{ax}}
\]
está se aproximando de zero à medida que $ x $ aumenta.
\end{proof}

Isso é útil ao calcular integrais impróprias que incluem polinômios e funções exponenciais.

